\chapter{Review et retrospective du premier sprint}

\section{Récapitulatif du projet}

Dans le cadre du cours de projet d'intégration, nous avons été amenés à choisir un produit à la fois innovant et représentatif de nos compétences acquises au cours des 3 dernières années. Ce produit doit répondre à la problématique posée par la  \textit{Maison du Développement Durable} de Louvain-la-neuve qui est la suivante : 
\begin{center}
\textbf{Comment réduire l'émission de CO² ?}
\end{center}

Après pas mal de discussions, nous avons décidé de viser les particuliers et leur consommation électrique.  \\


Notre produit, \textbf{Sharelet}, est une prise connectée. Elle communiquera avec une interface web, permettant une surveillance de la consommation et un contrôle distant de la prise. Les données seront centralisées, permettant une comparaison avec les autres utilisateurs!  \\ 

Avec les données, nous voudrions comparer et réaliser des statistiques pour que les cilents puissent se situer quant à leur consommation par rapport à d'autres. 

Nous souhaitons également avoir une communauté non silencieuse qui partagerait trucs et astuces pour consommer de manière plus raisonnable. 


\section{Avancement dans les Users Stories}
​La liste des User Stories terminées (selon la définition de "fini" que le groupe a choisi) et 
Le point sur ce qui a été réalisé par rapport à ce qui était prévu
\section{Estimation de la vélocité de l'équipe}
\begin{tabularx}{15cm}{|c|p{4.5cm}|X|}
    \hline
    \textbf{Catégorie} & \textbf{Membres} & \textbf{Remarques} \\
    \hline
    Electronique & Courtin Amélie, Delfosse Danielle, Surleraux Nicolas & commandes, soudure et tests des composants \\
    \hline
    Back-End & Surleraux Nicolas & mise en place des VMs de base de données, nat/firewall, autres\\
    \hline
    Developpement Arduino & Courtin Amélie, Mbarushimana Nadia, Surleraux Nicolas & developpement et flash de l'arduino \\ 
    \hline
    Developpement Web & Pettens Denis & developpement front et back end de l'API du site web \\
    \hline
    Business Model & Brancart Clément, Mbarushimana Nadia, Courtin Amélie & Mise en place du BMC \\
    \hline
    Suivi du projet & Delfosse Danielle & organisation des réunions et rapports \\ 
    \hline
\end{tabularx}
\section{Problématiques rencontrées}

\subsection{Le temps de livraison des différents composants} 

Le temps de livraison de certains composants tels que le module wifi et le relais ont pris plus de temps que prévu pour arriver dans notre boite au lettre, ce qui nous a bloqué dans notre avancement.
\subsection{La soudure très délicate} 
La soudure des pics sur l'Arduino nano et le module Wifi, des composants de petites tailles et assez fragiles, a été une étape difficile. Nous avons été voir un professionnel de la boutique Microtel à Wavre qui nous a gentimment soudé les élements. Malheureusement, nous avons perdu beaucoup de temps pour avoir tous les composants prêts au même instant.  

\subsection{Mauvaise direction} 
Nous pensions qu'il était possible de programmer le module wifi via l'arduino à l'aide d'une librairie spécifique. Malheureusement, nous n'avions pas pu tester avant d'avoir les composants soudés et nous avons constaté qu'il fallait directement travailler sur le module et non via l'Arduino. Nous avons alors du utiliser une roue de secours via un Raspberry Pi. 
 

\section{Conclusion et lancement du sprint 2}

Ce sprint, malgré une bonne organisation réalisé dès le début, a pris du retard à cause de certains éléments qui n'étaient pas de notre ressort expliqués ci-dessus dans les problématiques. 

Du fait du retard accumulé, nous avons dû revoir nos priorités dans les Users stories et repoussé ce qui était prévu à ce sprint qui était de petite priorité au sprint suivant. 





	
